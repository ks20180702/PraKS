\documentclass[12pt, a4paper, oneside, fontset=windows]{ctexart}
\usepackage{amsmath, amsthm, amssymb, appendix, bm, graphicx, hyperref, mathrsfs}

\title{\textbf{西洲曲}}
\author{佚名}
\date{\today}
\linespread{1.5}

\newtheorem{theorem}{定理}[section]
\newtheorem{definition}[theorem]{定义}
\newtheorem{lemma}[theorem]{引理}
\newtheorem{corollary}[theorem]{推论}
\newtheorem{example}[theorem]{例}
\newtheorem{proposition}[theorem]{命题}

\renewcommand{\abstractname}{\Large\textbf{摘要}}

\begin{document}
\pagestyle{empty}
\maketitle

\setcounter{page}{0}
\maketitle
\thispagestyle{empty}


\begin{abstract}
    忆梅下西洲,折梅寄江北。
    \par\textbf{关键词:}爱情; 诗歌.
\end{abstract}


\newpage
% 添加目录
\pagenumbering{Roman}
\setcounter{page}{1}
\tableofcontents
\newpage
\setcounter{page}{1}
\pagenumbering{arabic}

\newpage
\section{一级标题}

\subsection{二级标题}
这里是正文.

% 参考文献
\newpage

\begin{thebibliography}{99}
    \bibitem{a}作者. \emph{文献}[M]. 地点:出版社,年份.
    \bibitem{b}作者. \emph{文献}[M]. 地点:出版社,年份.
\end{thebibliography}


% 附录
\newpage

\begin{appendices}
    \renewcommand{\thesection}{\Alph{section}}
    \section{附录标题}
    这里是附录.
\end{appendices}

\end{document}